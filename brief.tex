\documentclass[12pt]{article}
\usepackage[utf8]{inputenc}
\usepackage{hyperref}

\title{Projet Passerelle 1 - Pendu}
\author{Mahrez BENDALI}
\date{28/12/2022}

\begin{document}
\maketitle

Bonjour et bienvenue dans ce document présentant ma réalisation dans le cadre du premier projet passerelle du programme Rocket, formation Margaret Hamilton.
Ce projet consistait à réimplémenter le célèbre jeu du pendu, dans lequel un mot mystère doit être trouvé, lettre par lettre, avec un nombre d'essais restreint.
Chaque bonne lettre trouvée dévoile toutes ses occurrences dans le mot mystère, et chaque mauvaise lettre rapproche de la défaite, représentée par un dessin de pendu.
Voici les fonctionnalités présentes :

\begin{itemize}
  \item proposer une lettre via le clavier physique
  \item proposer une lettre via les boutons affichés à l'écran
  \item proposer un mot entier
  \item ajouter un mot à la liste de mots possibles
  \item purger les mots ajoutés de la liste d'origine
\end{itemize}

Et voici les prochaines fonctionnalités sur lesquelles je compte travailler : 



créer une API dictionnaire, avec différents endpoints pour faire des thématiques de mots différentes (films, plantes, villes, animaux, Pokémon, etc)

gérer le multilangue dans les options, et traduire les instructions à l'écran ainsi que les listes de mots

améliorer le niveau de paramétrage du jeu (par exemple choix du nombre de vies)



\begin{itemize}
    \item améliorer le visuel du pendu
    \item améliorer l'expérience utilisateur (par ex. rendre le message de fin de partie plus joli)
    \item ajouter un mode sombre
    \item ajouter un compteur de victoires à la suite
    \item purger les mots ajoutés de la liste d'origine
    \item améliorer le niveau de paramétrage du jeu (par exemple choix du nombre de vies)
    \item gérer le multilangue dans les options, et traduire les instructions à l'écran ainsi que les listes de mots
    \item gérer des listes de mots avec des thématiques différentes (films, plantes, villes, animaux, Pokémon, etc)
  \end{itemize}

Le jeu est jouable gratuitement sur mon portfolio, trouvable à l'adresse suivante : \href{https://mahrezbendali.fr/projets/hangman.html}{https://mahrezbendali.fr/projets/hangman.html} !

\end{document}
